%插入样式内容
\documentclass[12pt]{article}
%%---------------------------------------------------------------------
% packages
% geometry
\usepackage{color}
\usepackage{geometry}
% font
\usepackage{fontspec}
\defaultfontfeatures{Mapping=tex-text}  %%如果没有它,会有一些 tex 特殊字符无法正常使用,比如连字符。
\usepackage{xunicode,xltxtra}
\usepackage[BoldFont,SlantFont,CJKnumber,CJKchecksingle]{xeCJK}  % \CJKnumber{12345}: 一万二千三百四十五
\usepackage{CJKfntef}  %%实现对汉字加点、下划线等。
\usepackage{pifont}  % \ding{}
% math
\usepackage{amsmath,amsfonts,amssymb}
% color
\usepackage{color}
\usepackage{xcolor}
\definecolor{EYE}{RGB}{199,237,204}
\definecolor{FLY}{RGB}{128,0,128}
\definecolor{ZHY}{RGB}{139,0,255}
% graphics
\usepackage[americaninductors,europeanresistors]{circuitikz}
\usepackage{tikz}
\usetikzlibrary{positioning,arrows,shadows,shapes,calc,mindmap,trees,backgrounds}  % placements=positioning
\usepackage{graphicx}  % \includegraphics[]{}
\usepackage{subfigure}  %%图形或表格并排排列
% table
\usepackage{colortbl,dcolumn}  %% 彩色表格
\usepackage{multirow}
\usepackage{multicol}
\usepackage{booktabs}
% code
\usepackage{fancyvrb}
\usepackage{listings}
% title
\usepackage{titlesec}
% head/foot
\usepackage{fancyhdr}
% ref
\usepackage{hyperref}
% pagecolor
\usepackage[pagecolor={EYE}]{pagecolor}
% tightly-packed lists
\usepackage{mdwlist}

\usepackage{styles/iplouccfg}
\usepackage{styles/zhfontcfg}
\usepackage{styles/iplouclistings}
\usepackage{listings} 
\usepackage{hyperref}
\usepackage{color,framed}
%\usepackage{styles/pythonhighlight}
%\usepackage{graphicx}
%%---------------------------------------------------------------------
% settings
% geometry
\geometry{left=2cm,right=1cm,top=2cm,bottom=2cm}  %设置 上、左、下、右 页边距
\linespread{1.5} %行间距
% font
\setCJKmainfont{Adobe Kaiti Std}
%\setmainfont[BoldFont=Adobe Garamond Pro Bold]{Apple Garamond}  % 英文字体
%\setmainfont[BoldFont=Adobe Garamond Pro Bold,SmallCapsFont=Apple Garamond,SmallCapsFeatures={Scale=0.7}]{Apple Garamond}  %%苹果字体没有SmallCaps
\setCJKmonofont{Adobe Fangsong Std}
% graphics
\graphicspath{{figures/}}
\tikzset{
    % Define standard arrow tip
    >=stealth',
    % Define style for boxes
    punkt/.style={
           rectangle,
           rounded corners,
           draw=black, very thick,
           text width=6.5em,
           minimum height=2em,
           text centered},
    % Define arrow style
    pil/.style={
           ->,
           thick,
           shorten <=2pt,
           shorten >=2pt,},
    % Define style for FlyZhyBall
    FlyZhyBall/.style={
      circle,
      minimum size=6mm,
      inner sep=0.5pt,
      ball color=red!50!blue,
      text=white,},
    % Define style for FlyZhyRectangle
    FlyZhyRectangle/.style={
      rectangle,
      rounded corners,
      minimum size=6mm,
      ball color=red!50!blue,
      text=white,},
    % Define style for zhyfly
    zhyfly/.style={
      rectangle,
      rounded corners,
      minimum size=6mm,
      ball color=red!25!blue,
      text=white,},
    % Define style for new rectangle
    nrectangle/.style={
      rectangle,
      draw=#1!50,
      fill=#1!20,
      minimum size=5mm,
      inner sep=0.1pt,}
}
\ctikzset{
  bipoles/length=.8cm
}
% code
\lstnewenvironment{VHDLcode}[1][]{%
  \lstset{
    basicstyle=\footnotesize\ttfamily\color{black},%
    columns=flexible,%
    framexleftmargin=.7mm,frame=shadowbox,%
    rulesepcolor=\color{blue},%
%    frame=single,%
    backgroundcolor=\color{yellow!20},%
    xleftmargin=1.2\fboxsep,%
    xrightmargin=.7\fboxsep,%
    numbers=left,numberstyle=\tiny\color{blue},%
    numberblanklines=false,numbersep=7pt,%
    language=VHDL%
    }\lstset{#1}}{}
\lstnewenvironment{VHDLmiddle}[1][]{%
  \lstset{
    basicstyle=\scriptsize\ttfamily\color{black},%
    columns=flexible,%
    framexleftmargin=.7mm,frame=shadowbox,%
    rulesepcolor=\color{blue},%
%    frame=single,%
    backgroundcolor=\color{yellow!20},%
    xleftmargin=1.2\fboxsep,%
    xrightmargin=.7\fboxsep,%
    numbers=left,numberstyle=\tiny\color{blue},%
    numberblanklines=false,numbersep=7pt,%
    language=VHDL%
    }\lstset{#1}}{}
\lstnewenvironment{VHDLsmall}[1][]{%
  \lstset{
    basicstyle=\tiny\ttfamily\color{black},%
    columns=flexible,%
    framexleftmargin=.7mm,frame=shadowbox,%
    rulesepcolor=\color{blue},%
%    frame=single,%
    backgroundcolor=\color{yellow!20},%
    xleftmargin=1.2\fboxsep,%
    xrightmargin=.7\fboxsep,%
    numbers=left,numberstyle=\tiny\color{blue},%
    numberblanklines=false,numbersep=7pt,%
    language=VHDL%
    }\lstset{#1}}{}
% pdf
\hypersetup{%pdfpagemode=FullScreen,%
            pdfauthor={Haiyong Zheng},%
            pdftitle={Title},%
            CJKbookmarks=true,%
            bookmarksnumbered=true,%
            bookmarksopen=false,%
            plainpages=false,%
            colorlinks=true,%
            citecolor=green,%
            filecolor=magenta,%
            linkcolor=cyan,%red(default)
            urlcolor=cyan}
% section
%http://tex.stackexchange.com/questions/34288/how-to-place-a-shaded-box-around-a-section-label-and-name
\newcommand\titlebar{%
\tikz[baseline,trim left=3.1cm,trim right=3cm] {
    \fill [cyan!25] (2.5cm,-1ex) rectangle (\textwidth+3.1cm,2.5ex);
    \node [
        fill=cyan!60!white,
        anchor= base east,
        rounded rectangle,
        minimum height=3.5ex] at (3cm,0) {
        \textbf{\thesection.}
    };
}%
}
\titleformat{\section}{\Large\bf\color{blue}}{\titlebar}{0.1cm}{}
% head/foot
\setlength{\headheight}{15pt}
\pagestyle{fancy}
\fancyhf{}

\chead{\color{black!50!green}Cifar10 And MNIST}

%\lfoot{\color{blue!50!green}Dai Jialun}
\cfoot{\color{blue!50!green}\href{http://vision.ouc.edu.cn/~zhenghaiyong}{CVBIOUC}}
\rfoot{\color{blue!50!green}$\cdot$\ \thepage\ $\cdot$}
\renewcommand{\headrulewidth}{0.4pt}
\renewcommand{\footrulewidth}{0.4pt}


%%---------------------------------------------------------------------
\begin{document}
%%---------------------------------------------------------------------
%%---------------------------------------------------------------------
% \titlepage
\title{\vspace{-2em} Cifar-10与MNIST训练集\\
\normalsize{}}
\author{Dai Jialun}
\date{\vspace{-0.7em} \today \vspace{-0.7em}}
%%---------------------------------------------------------------------
\maketitle\thispagestyle{fancy}
%%---------------------------------------------------------------------
\maketitle
%\tableofcontents 
\section{Cifar10}
CIFAR-10数据集由10类总共60000张32$\times$32的彩色图像组成,其中每一类都有6000张图像。在60000张图像中,有50000张的训练图像与10000张的测试图像。

数据集被分成5个训练组和1个测试组,每一组都有10000张图像。测试组中包含了从每一类中随机挑选出的1000张图像。训练组从所剩下的图像中随机挑选,因此在某些训练组中,某一类中的图像可能比其他类图像多。另外从整体上看,5组训练组包含了每一类的5000张图像。

数据集中,每一类的图像都是独立存在,没有相互包含的情况。

用cuda-convnet实现的卷积神经网络,其系统在没有数据增强的情况下,测试错误为18\%;有数据增强的情况下,测试错误为11\%。

数据集的布局:

Python/Matlab版:在这里介绍了数据集的Python版本。Matlab版本的布局与其是相同的。

在下载的文档中,包含了{\scriptsize{data\_batch\_1}},{\scriptsize{data\_batch\_2}},{\scriptsize{data\_batch\_3}},{\scriptsize{data\_batch\_4}},{\scriptsize{data\_batch\_5}},以及{\scriptsize{test\_batch}}。每一个文件都是由Python的``cPickle''所产生的pickled对象。下面是Python程序,会打开一个文件并且返回一个目录。

\begin{python}
def unpickle(file)
  Import cPickle
  fo=open(file,'rb')
  dict=cPickle.load(fo)
  fo.close()
  return dict
\end{python} 

通过以上方式加载后,每一个组文件将会生成一个包含以下文件的文件夹:
\begin{description}
\item[data] 一个类型为unit8的10000$\times$3072的numpy矩阵。这个矩阵的每一行存储一个32$\times$32的彩色图像,刚开始的1024个数(32$\times$32)表示红色通道的值,接下来的1024个数(32$\times$32)表示绿色通道,最后的1024个数(32$\times$32)表示蓝色通道。图像是行优先存储的,因此数组刚开始的32是第一行图像的红色通道值。
\item[labels] 一个列表包含10000个数字,范围为0~9。在目录中,第i个数字表示在数组数据中,第i个图像的标注。
\end{description}

数据集包含了另外一个{\scriptsize{batches.meta}}文件。它也包含了一个Python目录对象,有以下内容:
\begin{description}
\item[label\_names] 一个含有10个目标的列表,对于上述标注数组,其给了数字标注有意义的名字。例如,label\_names[0]=``airplane''
\end{description}

二进制版本的数据集(cifar-10-batches-bin)

二进制版本包含了{\scriptsize{data\_batch\_1.bin}},{\scriptsize{data\_batch\_2.bin}},{\scriptsize{data\_batch\_3.bin}},{\scriptsize{data\_batch\_4.bin}},{\scriptsize{data\_batch\_5.bin}},以及{\scriptsize{test\_batch\_bin}}。每个文件的格式如下:

\colorbox{gray}{  
\vbox{  \scriptsize{
<1 $\times$ label> <3072 $\times$ pixel>

$\dots$

<1 $\times$ label> <3072 $\times$ pixel>  }
}}  

换句话说,刚开始的比特是第一张图像的标注,是一个范围在0~9之间的数字。接下来的3072比特是图像像素值。开始的1024个数(32$\times$32)是红色通道的值,接下来的1024个数(32$\times$32)是绿色通道的值,最后的1024个数(32$\times$32)是蓝色通道的值。这些值是按行优先排列的,因此最初的32比特是图像的第一行的红色通道值。

每一个文件包含10000个这样3073个比特的图像``行'',尽管在文件中没有特别划分这些行的界限。因此每个文件都应该有30730000比特这么长。

另外,还有一个 {\scriptsize{batches.meta.txt}}的文件。这是一个ASCII文件,表示每一个类别名字各自匹配范围在0~9之间的数字。这仅仅是一个10类别名字的列表,每一个一行。在第i行的类别名对应数字标号i。


\section{MNIST}
MNIST数据集是手写数字图像,其训练集有60000张图像,测试集有10000张图像。这只是一个更大的数据集NIST的一个子集。这些数字图像已经经过尺寸的正规化与在固定大小图像中的置中心处理。

这个数据集适合一些人想用学习技术以及模式识别方法来处理真实世界的数据,而且不需要花太多功夫在预处理和格式化方面的工作上。

这个数据集总共有4个文件:
\begin{description}
\item[{\footnotesize{train-images-idx3-ubyte.gz}}] {\footnotesize{training set images (9912422 bytes)}} 
\item[{\footnotesize{train-labels-idx1-ubyte.gz}}] {\footnotesize{training set labels (28881 bytes)}} 
\item[{\footnotesize{t10k-images-idx3-ubyte.gz}}] {\footnotesize{test set images (1648877 bytes)}} 
\item[{\footnotesize{t10k-labels-idx1-ubyte.gz}}] {\footnotesize{test set labels (4542 bytes)}}
\end{description}

在NIST中原始的黑白图像,为了保持长宽比,图像经过尺寸正规化处理变为20$\times$20的像素框。结果图像是灰度保真图,使用了规范化算法处理。而且,图像是在28$\times$28的图像中心位置。

MNIST数据集是由NIST的Special Database 3 和Special Database 1所组成,二者包含了手写数字的二值图像。在NIST中,它将SD-3作为训练集,SD-1作为测试集。然而,SD-3比SD-1更清洗,更容易识别。理由可能是因为SD-3是从Census Bureau的员工处收集的,而Sd-1是从高中生处收集的。从学习实验中得出可靠的结论要求结果与训练集的选择无关,并且测试样本的完整数据集。因此,必须通过混合NIST数据集来建立一个全新的数据集。‘

MNIST训练集是由SD-3的30000张图像和SD-1的300000张图像组成。测试集是由SD-3的5000张图像和SD-1的5000张图像组成。60000张的训练集所包含的例子从大约250个人中收集到的。因此,我们可以确定训练集与测试集的数字是不同的人所写的。

SD-1是由500个不同的人所写的58527张数字图像。与SD-3不同,在SD-3中每个人所写的数字图像按序列排放,而在SD-1中,数据是杂乱无章的。人工分辨SD-1是可行的,我们用这些信息来分开所写人。我们将SD-1分为两部分:由前250位个人的数字作为型的训练集,剩下的250个人所写数字放在测试集中。因此,我们有了两个子集,每个自己有大约30000张图像。新的训练集从SD-3中的\#0开始抽取足够多的图像,形成一个60000张的训练图像。与其相似,新的测试集是由SD-3中的\#35000开始,形成一个60000的测试图像。只有测试集中的10000张测试图像(从SD-1中的5000张图像和从SD-3中的5000张图像)在这里是可用。当然,60000张的训练图像也是可用的。

MNIST数据库的文件格式:

数据存储在一个非常简单的用来存放向量和多维矩阵的文件格式中。在文件中所有的整数都
%%---------------------------------------------------------------------
\end{document}
